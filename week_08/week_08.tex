% Options for packages loaded elsewhere
\PassOptionsToPackage{unicode}{hyperref}
\PassOptionsToPackage{hyphens}{url}
%
\documentclass[
]{article}
\usepackage{amsmath,amssymb}
\usepackage{iftex}
\ifPDFTeX
  \usepackage[T1]{fontenc}
  \usepackage[utf8]{inputenc}
  \usepackage{textcomp} % provide euro and other symbols
\else % if luatex or xetex
  \usepackage{unicode-math} % this also loads fontspec
  \defaultfontfeatures{Scale=MatchLowercase}
  \defaultfontfeatures[\rmfamily]{Ligatures=TeX,Scale=1}
\fi
\usepackage{lmodern}
\ifPDFTeX\else
  % xetex/luatex font selection
\fi
% Use upquote if available, for straight quotes in verbatim environments
\IfFileExists{upquote.sty}{\usepackage{upquote}}{}
\IfFileExists{microtype.sty}{% use microtype if available
  \usepackage[]{microtype}
  \UseMicrotypeSet[protrusion]{basicmath} % disable protrusion for tt fonts
}{}
\makeatletter
\@ifundefined{KOMAClassName}{% if non-KOMA class
  \IfFileExists{parskip.sty}{%
    \usepackage{parskip}
  }{% else
    \setlength{\parindent}{0pt}
    \setlength{\parskip}{6pt plus 2pt minus 1pt}}
}{% if KOMA class
  \KOMAoptions{parskip=half}}
\makeatother
\usepackage{xcolor}
\usepackage[margin=1in]{geometry}
\usepackage{graphicx}
\makeatletter
\def\maxwidth{\ifdim\Gin@nat@width>\linewidth\linewidth\else\Gin@nat@width\fi}
\def\maxheight{\ifdim\Gin@nat@height>\textheight\textheight\else\Gin@nat@height\fi}
\makeatother
% Scale images if necessary, so that they will not overflow the page
% margins by default, and it is still possible to overwrite the defaults
% using explicit options in \includegraphics[width, height, ...]{}
\setkeys{Gin}{width=\maxwidth,height=\maxheight,keepaspectratio}
% Set default figure placement to htbp
\makeatletter
\def\fps@figure{htbp}
\makeatother
\setlength{\emergencystretch}{3em} % prevent overfull lines
\providecommand{\tightlist}{%
  \setlength{\itemsep}{0pt}\setlength{\parskip}{0pt}}
\setcounter{secnumdepth}{-\maxdimen} % remove section numbering
\ifLuaTeX
  \usepackage{selnolig}  % disable illegal ligatures
\fi
\IfFileExists{bookmark.sty}{\usepackage{bookmark}}{\usepackage{hyperref}}
\IfFileExists{xurl.sty}{\usepackage{xurl}}{} % add URL line breaks if available
\urlstyle{same}
\hypersetup{
  pdftitle={Summaries week 8},
  pdfauthor={Meritxell Feliu Ribas},
  hidelinks,
  pdfcreator={LaTeX via pandoc}}

\title{Summaries week 8}
\usepackage{etoolbox}
\makeatletter
\providecommand{\subtitle}[1]{% add subtitle to \maketitle
  \apptocmd{\@title}{\par {\large #1 \par}}{}{}
}
\makeatother
\subtitle{Phonetics and Phonology of Bilingualism}
\author{Meritxell Feliu Ribas}
\date{2024-03-15}

\begin{document}
\maketitle

\hypertarget{reading-1-bradlow-1995}{%
\section{Reading 1: Bradlow (1995)}\label{reading-1-bradlow-1995}}

Bradlow (1995) compared the acoustic vowel systems of General American
English and Madrid Spanish (plus Greek) in order to examine the effect
of inventory size, as English vowel system is larger than that of
Spanish (and Greek). To that end, she analyzed vowel formant
measurements (F1 and F2) from vowels that occurred between voiceless
plosives (/p,t,k/), elicited from 8 participants (4 English and 4
Spanish speakers, respectively) in a read aloud task. Whereas English
vowels had higher F2 values, indicating a different tongue position,
there were no significant differences between languages as for F1
values. Data from Spanish were compared to Greek (both have 5
monophthongs), showing that Spanish vowels had generally higher F2 and
lower F1 values. In addition, the acoustic vowel space of English had a
larger area, but it was dependent on the syllabic context, and there
were no significant differences between point and nonpoint vowels in any
of the tested languages. These results support the idea of
language-specific constraints in the production of vowels and their
acoustic space (language-specific base-of-articulation property).

\textbf{Reference:} Bradlow, A. R. (1995). ``A comparative acoustic
study of english and spanish vowels''. In: \emph{The Journal of the
Acoustical Society of America} 97.3, pp.~1916--1924.

\hypertarget{reading-2-huxf8jen-and-flege-2006}{%
\section{Reading 2: Højen and Flege
(2006)}\label{reading-2-huxf8jen-and-flege-2006}}

I am the discussion leader for this article.

\textbf{Reference:} Højen, A. and J. E. Flege (2006). ``Early learners'
discrimination of second-language vowels''. In: \emph{The Journal of the
Acoustical Society of America} 119.5, pp.~3072--3084.

\hypertarget{reading-3-celdruxe1n-and-elvira-garcuxeda-2019}{%
\section{Reading 3: Celdrán and Elvira-García
(2019)}\label{reading-3-celdruxe1n-and-elvira-garcuxeda-2019}}

This is a chapter from the book \emph{Key Issues in the Teaching of
Spanish Pronunciation}. According to the syllabus, no summary is
required for chapters.

\textbf{Reference:} Celdrán, E. M. and W. Elvira-Garcia (2019).
``Description of Spanish Vowels and Guidelines for teaching them''.In:
\emph{Key Issues in the Teaching of Spanish Pronunciation}. Ed. by R.
Rao. Routledge, pp.~17--39.

\hypertarget{notes-on-huxf8jen-and-flege-2006}{%
\section{Notes on Højen and Flege
(2006)}\label{notes-on-huxf8jen-and-flege-2006}}

\begin{itemize}
\item
  General topic: Perception of L2 sounds
\item
  More specific: The discrimination of English vowels by L1 Spanish - L2
  English early bilinguals (learners*)

  \begin{itemize}
  \tightlist
  \item
    Me parece curioso el hecho de que los llamen ``early learners'' y no
    ``early bilinguals''. Sé que es solo terminología, pero
    bueno\ldots{}
  \item
    Asimismo, llaman a los ML de la L2, L2 native speakers, which also
    confusing for me.
  \end{itemize}
\item
  Background lit.:

  \begin{itemize}
  \tightlist
  \item
    Infants begin to perceive speech in a languge-specific way during
    the first year of life, when they develop perceptual processes
    specialized for the ambient language (when they are born, they can
    discriminate all sounds, from all languages) -\textgreater{} Shift
    from a language-general to a language-specific pattern of perception
  \item
    It has been proposed that, as people grow older, L2 learning is
    hindered by a loss of neuroplasticity that accompanies normal neural
    maturation -\textgreater{} L2 speech acquisition is impeded by the
    acquisition of the L1 phonological system.
  \item
    Native Language Magnet theory (NLM theory):

    \begin{itemize}
    \tightlist
    \item
      tries to explain what exactly happens (see bullet points above)
      and why
    \item
      holds that\ldots{}

      \begin{itemize}
      \tightlist
      \item
        infants categorize sound patterns into a ``perceptual map''
        (``sound map'')
      \item
        babies create perfect examples of sounds with a target area
        around each sound
      \item
        these prototypes ``tune'' the child's brain to the native
        language (statistical learning?)
      \item
        once a sound category exists in memory, ``it functions like a
        magnet for other sounds'' (Kuhl, 2000, p.~11853). That is, the
        prototype attracts sounds that are similar so that they sound
        like the prototype itself
      \item
        thus, the identification of L1 speech sounds become more robust;
        L2 categories may become difficult (L2 learning as more
        challenging)
      \end{itemize}
    \item
      describes how innate factors and early experience with language
      interact in the development of speech perception
    \item
      Some evidence suggest that this ``sound map'' may be difficult to
      reverse (L1 Spanish - L2 Catalan), whereas others point to the
      opposite direction (L1 Italian - L2 Canadian English)
      -\textgreater{} Conflicting results: L1 use, testing techniques,
      age of participants, type of input (L2 accented or not), etc.

      \begin{itemize}
      \tightlist
      \item
        ADD THIS AS A QUESTION; E.G., CAN THIS SOUND MAP BE REVERSED
        BASED ON THE STUDIES DISCUSSED IN THE LIT REVIEW SECTION?
      \end{itemize}
    \end{itemize}
  \item
    Bosch et al.~(2000), Pallier et al.~(1997, 2001), Sebastián-Gallés
    and Soto-Faraco (1999), Sebastián-Gallés et al.~(2005)

    \begin{itemize}
    \tightlist
    \item
      Early learners obtained significantly lower scores than Catalan
      native speakers, and showed a wider range of scores (/e/ vs /ɛ/)
    \item
      Main conclusions:
    \item
      ``the speech perception system doesn't seem able to easily develop
      new phonetic categories after perceptual attunement to the L1
      phonological system''
    \item
      ``early learners exhibit a lack of behavioral plasticity with
      respect to learning to perceive contrastive phonemes of an L2''
    \item
      ``there are severe limitations to the malleability of the
      initially acquired L1 phonemic categories even under conditions of
      early and extensive exposure''
    \item
      SO\ldots{} early learners are unlikely to perceive L2 vowels in a
      nativelike way
    \end{itemize}
  \item
    Flege et al.~(1999), Flege and MacKay (2004)

    \begin{itemize}
    \tightlist
    \item
      Low-L1-use group and English ML got similar results in a
      discrimination task of English vowels; high-L1-use group differed
      marginally from MLs.
    \item
      Main conclusions:
    \item
      ``early exposure to the L2 does not guarantee the formation of new
      phonetic categories for L2 vowels, but that category formation
      will be more likely for individuals who use their L1
      infrequently''
    \item
      SO\ldots{} early learners may closely resemble L2 native speakers,
      if they use their L1 infrequently
    \end{itemize}
  \item
    A raíz de estos conflicting results, surge the present study! :)
  \end{itemize}
\item
  The present study:

  \begin{itemize}
  \tightlist
  \item
    Previous research showed two different trends/patterns:

    \begin{enumerate}
    \def\labelenumi{(\arabic{enumi})}
    \tightlist
    \item
      early learners differed significantly from the control group (L2
      native speakers), SO\ldots{} they are unlikely to perceive L2
      vowels in a nativelike way
    \item
      early learners did not significantly differ from the control group
      (L2 native speakers), SO\ldots{} they may closely resemble L2
      native speakers, if they use their L1 infrequently
    \end{enumerate}
  \item
    About Højen and Flege (2006)

    \begin{itemize}
    \tightlist
    \item
      Goal: to examine the discrimination of English vowels by L1
      Spanish - L2 English early bilinguals (learners*)
    \item
      But\ldots{} What is the novelty of this study?

      \begin{itemize}
      \tightlist
      \item
        They assessed early learners' perceptual learning by comparing
        their performance to that of 2 control groups: English MLs and
        Spanish MLs
      \end{itemize}
    \item
      And what is the reasoning behind this change in methods?

      \begin{itemize}
      \tightlist
      \item
        English MLs -\textgreater{} to know if their perceptual learning
        is complete
      \item
        Spanish MLs -\textgreater{} to determine how much they had
        learned, even in the absence of complete learning
      \end{itemize}
    \item
      2 experiments
    \end{itemize}
  \end{itemize}
\item
  Experiment 1:

  \begin{itemize}
  \item
    Aim: to develop an AXB test that would result in higher
    (near-ceiling) scores for vowels heard as distinct phonemes and
    lower (near-chance) scores for foreign vowels that might be heard as
    belonging to the same category in the L1.

    \begin{itemize}
    \tightlist
    \item
      e.g., in Catalan, for L1 Spanish: {[}u{]} - {[}o{]}; {[}e{]} vs
      {[}ɛ{]}
    \item
      Why is it important?

      \begin{itemize}
      \tightlist
      \item
        para asegurarse de que no lo acierten ``a potra'', es decir,
        miran cómo hacer la tarea más difícil para evitar que los
        resultados estén hinchados por hacerlo a boleo -\textgreater{}
        to better assess L2 vowel discrimination abilities
      \item
        it often produces very high discrimination scores (90\%-100\%
        correct) for pairs of foreign vowels that are likely perceived
        as instances of a single phonetic category in the listener's
        native language (L1)
      \item
        if the Spanish monolingual group achieves near-ceiling scores
        due to non-phonetic encoding strategies (above change can be due
        to context codes -\textgreater{} the listener positions each
        stimulus in relation to the full range of stimuli presented in
        the experiment), they cannot effectively serve as a baseline
      \item
        3 different processess in the discrimination of speech sounds as
        wiether wihtin- or between-category contrasts might be used:
        labeling, context coding, trace coding
      \end{itemize}
    \end{itemize}
  \item
    What is an AXB test?

    \begin{itemize}
    \tightlist
    \item
      a type of perceptual discrimination task
    \item
      participants listen to three stimuli: A, X, and B
    \item
      they determine whether X sounds more like A or B
    \end{itemize}
  \item
    Participants: 45 English monolinguals; 3 native speakers of Danish
  \item
    Stimuli: nonwords (CCVC) with 4 vowel contrasts inserted:

    \begin{itemize}
    \tightlist
    \item
      easy contrast: {[}{]} - {[}{]} -\textgreater{} /gl\_s/
    \item
      difficult contrasts:

      \begin{itemize}
      \tightlist
      \item
        {[}{]} - {[}{]} -\textgreater{} /fl\_s/
      \item
        {[}{]} - {[}{]} -\textgreater{} /kl\_s/
      \item
        {[}{]} - {[}{]} -\textgreater{} /tj\_s/
      \end{itemize}
    \item
      3 conditions:

      \begin{itemize}
      \tightlist
      \item
        blocked: contrasts were presented in separate blocks (small
        stimulus range) -\textgreater{} All stimuli in declarative
        sentences: ``Det hed virkelig \_\_ engang''
      \item
        mixed: contrasts were randomly presented in one large block
        (large stimulus range) -\textgreater{} All stimuli in
        declarative sentences: ``Det hed virkelig \_\_ engang''
      \item
        mixed/F0: one large block + variation in F0 (high vs low F0):

        \begin{itemize}
        \tightlist
        \item
          Half stimuli in declarative sentences (``Det hed virkelig \_\_
          engang'') and the other half in interrogative sentences (``Hed
          det virkelig \_\_ engang?'')
        \item
          La variación en el pitch se consigue con el uso de declarative
          vs interrogative sentences (lower vs higher pitch,
          respectively)
        \item
          Inclusion of different F0 patterns:

          \begin{itemize}
          \tightlist
          \item
            no F0 change (HHH; LLL)
          \item
            neutral trials (HLH; LHL) -\textgreater{} f0 of X (target)
            is different than its match and mismatch
          \item
            congruent trials (HHL; LLH; HLL; LHH) -\textgreater{} f0 of
            X is the same as the f0 of its ``match''
          \item
            incongruent trials -\textgreater{} f0 of X is different than
            the f0 of its ``match''
          \end{itemize}
        \end{itemize}
      \end{itemize}
    \end{itemize}
  \item
    Procedure:

    \begin{itemize}
    \tightlist
    \item
      64 trials x 4 contrasts = 256 AXB trials
    \item
      3 stimuli
    \item
      push 1-2 or 2-3 button (1-2 if 2 was like 1, 2-3 if 2 was like 3)
    \item
      practice block with feedback
    \end{itemize}
  \item
    Results:

    \begin{itemize}
    \tightlist
    \item
      Danish MLs had higher scores in all conditions (easy and
      difficult)
    \item
      English MLs had higher scores in the easy condition, and
      substantially lower scores in the difficult conditions, but still
      above chance. However, they were lowered in the mixed/F0 condition
      compared to the blocked condition (and especially lowered in f0
      incongruent trials; for neutral was inbetween congruent and
      incongruent). Interestingly, there were no significant differences
      between the blocked and the mixed conditions, which suggests that
      the differences between the blocked and the mixed/f0 was due to F0
      manipulation. So, f0 seemed to have an effect on lowering the
      scores (``more difficult'').
    \item
      HENCE, they decided to use the mixed/f0 condition for experiment
      2, considering only incongruent (target different as matched) and
      neutral f0 (target different as both matched and mismatched)
      types.
    \end{itemize}
  \end{itemize}
\item
  What is the difference between a CDT and a AXB test? (1) Stimulus
  presentation: * In a CDT, participants typically hear two stimuli
  sequentially and are asked to determine whether they are the same or
  different. The stimuli are usually from different phonetic categories.
  * In an AXB test, participants hear three stimuli sequentially
  (labeled A, X, and B) and are asked to determine whether X sounds more
  like A or B. The X stimulus may be from the same category as A or B,
  and the task is to discriminate between the two categories. (2) Task
  complexity: * CDTs tend to be simpler tasks because participants only
  need to judge whether two stimuli are the same or different. \emph{AXB
  tests can be more complex because participants must not only
  discriminate between stimuli but also compare them to two reference
  stimuli (A and B) to make their judgment. (3) Purpose: } CDTs are
  often used to measure sensitivity to phonetic contrasts and
  categorical perception, i.e., how individuals categorize speech sounds
  into distinct phonetic categories. * AXB tests are used to assess
  discrimination abilities between speech sounds, particularly when
  stimuli are drawn from similar phonetic categories or when the task
  involves categorical perception.

  \begin{itemize}
  \tightlist
  \item
    Experiment 2:

    \begin{itemize}
    \tightlist
    \item
      Aim:

      \begin{itemize}
      \tightlist
      \item
        to analyze the discrimination of English vowels by adult L1
        Spanish - L2 early learners of English and compare it to that of
        English monolinguals (and Spanish monolinguals)
      \end{itemize}
    \item
      Methodology:

      \begin{itemize}
      \tightlist
      \item
        Participants (n=60):

        \begin{itemize}
        \tightlist
        \item
          20 L1 Spanish - L2 English early bilinguals
        \item
          20 English monolinguals
        \item
          20 Spanish monolinguals
        \end{itemize}
      \item
        Materials:

        \begin{itemize}
        \tightlist
        \item
          (Language background questionnaire)
        \item
          (Production task)
        \item
          Perception task
        \end{itemize}
      \item
        Stimuli:

        \begin{itemize}
        \tightlist
        \item
          English nonwords (CCVC) with 4 vowel contrasts:

          \begin{itemize}
          \tightlist
          \item
            easy contrast: {[}i{]} - {[}u{]} -\textgreater{} /gl\_s/
          \item
            difficult contrasts:

            \begin{itemize}
            \tightlist
            \item
              {[}ʊ{]} - {[}oʊ{]}* -\textgreater{} /fl\_s/
            \item
              {[}ɪ{]} - {[}eɪ{]}* -\textgreater{} /kl\_s/
            \item
              {[}ɑ{]} - {[}ʌ{]} -\textgreater{} /tʃ\_s/
            \end{itemize}
          \end{itemize}
        \item
          Foto (fig.~2)
        \item
          Declarative (low-f0) and interrogative sentences (high-f0)
        \item
          Neutral f0 type and incongruent f0 type
        \end{itemize}
      \end{itemize}
    \item
      Procedure:

      \begin{itemize}
      \tightlist
      \item
        64 trials x 4 contrasts = one block of 256 trials
      \item
        ISI of 0 or 1000 ms -\textgreater{} ``the block containing all
        256 trials was presented two times in counterbalanced order with
        an ISI of either 0 or 1000 ms''
      \item
        3 stimuli
      \item
        push 1-2 or 2-3 button (1-2 if 2 was like 1, 2-3 if 2 was like
        3)
      \item
        practice block with feedback
      \end{itemize}
    \item
      Results:

      \begin{itemize}
      \tightlist
      \item
        Overall, scores were significantly lower in the incongruent f0
        condition - similar effect across the 3 groups
      \item
        Overall, scores were lower, on average, for ISI=0 ms than
        ISI=1000 ms

        \begin{itemize}
        \tightlist
        \item
          English MLs scored near ceiling for the difficult contrasts
        \item
          Spanish MLs scored near chance for the difficult contrasts
        \item
          BLs obtained significantly higher scores than Spanish MLs at
          both ISIs
        \item
          BLs didn't differ from English MLs for any contrast at
          ISI=1000 ms
        \item
          BLs scored lower than English MLs for two difficult contrasts
          at ISI=0 ms (suggesting that long-term memory representations
          for English vowels were not identical to the English MLs')
        \item
          Scores obtained for individual BLs and Spanish MLs for the
          three difficult contrasts were non-overlapping at ISI=0ms,
          which provides evidence of extensive perceptual learning in
          early learners -\textgreater{} It is inconsistent with the
          view that speech perception system loses plasticity and that
          new phonetic categories become unlikely following attunement
          to the L1 sound system
        \end{itemize}
      \item
        Easy contrast: higher scores than in the difficult contrast,
        but\ldots{}

        \begin{itemize}
        \tightlist
        \item
          BLs' scores were slightly lower than those of English MLs
        \item
          BLs' scores were significantly higher than those of Spanish
          MLs
        \end{itemize}
      \item
        Difficult contrast:

        \begin{itemize}
        \tightlist
        \item
          Spanish MLs' scores were all near chance
        \item
          BLs' scores were lowered than those of English MLs in
          {[}{]}-{[}{]} and {[}{]}-{[}{]}. Importantly, these scores
          were more similar to those of English MLs than those of
          Spanish MLs
        \end{itemize}
      \item
        Individual differences among BLs:

        \begin{itemize}
        \tightlist
        \item
          3 of the lowest-scoring BLs had a later AoE, compared to the
          highest-scoring BLs -\textgreater{} Age of Exposure
        \item
          3 of the highest-scoring BLs reported using English more with
          friends in the past 5 years and during the first 5 years of
          exposure to English, compared to the lowest-scoring BLs
          -\textgreater{} L2 use
        \item
          3 of the 4 highest-scoring BLs reported having attended school
          in the U.S. longer, compared to the lowest-scoring BLs
          -\textgreater{} L2 exposure
        \item
          Differences in AoE and L2 might be responsible for important
          individual differences among BLs
        \end{itemize}
      \end{itemize}
    \item
      Discussion:

      \begin{itemize}
      \tightlist
      \item
        The vowel pair in the easy contrast was heard as distinct
        Spanish vowels for Spanish MLs
      \item
        Each vowel pair in the difficult contrast was heard as a single
        Spanish vowel category for Spanish MLs -\textgreater{} They
        could not discriminate between them, although they were
        different in English
      \item
        Important individual differences among BLs -\textgreater{} No es
        un grupo homogéneo!
      \item
        Factors that may influence L2 speech perception: amount of L2
        input, type of L2 input, L2 use, age of first exposure to the L2
        -\textgreater{} Age effects might exist
      \item
        BLs can show nativelike discrimination of English vowel
        contrasts
      \item
        However, some BLs obtained scores below English ML
        -\textgreater{} Early exposure to an L2 doesn't guarantee
        nativelike discrimination of L2 phonetic segments
        -\textgreater{} There might be other factors that influence such
        perception
      \end{itemize}
    \end{itemize}
  \item
    Potential questions:

    \begin{itemize}
    \tightlist
    \item
      What kind of studies are described in the background literature?
      How do they inform the present study?
    \item
      What have previous studies on the perception of word stress found?
    \item
      What is the topic of analysis?
    \item
      Why do they focus on the vowels {[}a{]} and {[}i{]}? Why
      unaccented contexts?
    \item
      Why did the authors choose to manipulate duration, overall
      intensity, and spectral tilt as potential cues to stress?
    \item
      What might explain the differential perception of stress contrast
      depending on vowel type ({[}a{]} vs.~{[}i{]})?
    \item
      What are your main takeaways of this study? Why is it important?
      How do the findings in this study contribute to our knowledge of
      L2 perceptual abilities?
    \end{itemize}
  \item
    What are the implications of these findings for both teaching and
    learning?
  \item
    How might future research build upon these findings to further
    investigate perception of L2 sounds (Spanish, English, other
    languages)?
  \end{itemize}
\end{itemize}

\end{document}
